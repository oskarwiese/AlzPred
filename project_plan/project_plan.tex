\documentclass[]{article}

%opening
\title{Project Plan}
\author{Oskar Wiese and Anders Henriksen}

\begin{document}

\maketitle
\noindent
Alzheimer's Disease (AD) is a disease which in the year 2021 affected an estimated 5.8 million people aged 65 years or older in America alone.
By the time an individual shows early symptoms of AD, whether by a screening test or a physical measurement of the brain volume, it is already too late to prevent the rapid development of the disease. 
Thus, to decrease cost of treatment and increase chance of successful rehabilitation, it is becoming ever increasingly important to detect AD at earlier stages than what is accomplishable by the current state of the art.
Therefore, we wish to explore whether tools of computer vision and machine learning allows for quicker detection of AD, before an individual shows symptoms detectable by the common screening process of doctors. 
Many studies has been conducted within the field of alzheimer, and all of the studies including magnetic resonance imaging (MRI) scans are stored in the Alzheimers's Disease Neuroimaging Initiative (ADNI).
The MRI scans range from an electromagnetic field of 1.5 tesla (T) to 3T. The quality of the 3T images is higher and the price follows this same trend, meaning that MRI electromagnetic field strength is determined mostly by funding.
This, inherently, will introduce a bias in an algorithm trained on a dataset with MRI images spanning both magnetic field strengths, resulting in images with different levels of detail.

The scope of this project will be to train a classifier on MRI images to predict AD as well as training a cycle-generative adversarial network (cycleGAN) to construct 1.5T images from 3T images in order to have more usable, and hopefully unbiased, data. 
Thereby, the model will have more available training data leading to a more successful classifier which will be able to aid doctors and humans in need all around the world.
This project will also aim to determine possible bias introduced in the model, if any, and discuss how a classifier can be implemented as a tool for doctors, how this might benefit AD prevention and treatment cost and the possible ethical scenarios that might be at play. 

\section{Research Questions}
Research Question: Is it possible with tools of ML to predict Alzheimers at an earlier stage than previously possible?
\begin{itemize}
	\item How can a cycle-GAN be used to convert between 1.5T and 3T MRI images? Does the data prove useful for classification? (Based on preliminary studies it is found that 1.5T and 3T respectively create noise. Can we then try to find a way to compensate for this by generating 1.5T from 3T and reverse using a CycleGan)
	\item How effective is the prediction model in predicting whether a patient has alzheimer's disease? ( Possibility of further developing the current model with more specific methods )
	\item How will the model change the future? (Societal and economic impacts given the success of the prediction model?)
\end{itemize}

\end{document}
